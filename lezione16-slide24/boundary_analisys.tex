\documentclass[a4paper, 12pt]{article}

\usepackage[T1]{fontenc}

\usepackage{lipsum}
\usepackage{pifont}
\usepackage{amssymb}
\usepackage{amsmath}
\usepackage[utf8]{inputenc}
\usepackage[italian]{babel}
\usepackage{graphicx}
\graphicspath{ {./images/} }
\usepackage{spverbatim}
\usepackage{float}
\usepackage{url}
\usepackage{xstring}
\usepackage{hyperref}
\usepackage{xcolor}
\definecolor{linkcolor}{RGB}{1,1,87}
\hypersetup{
    colorlinks,
    citecolor=black,
    filecolor=black,
    linkcolor=linkcolor,
    urlcolor=blue
}
\usepackage{fancyvrb,newverbs,xcolor}
\definecolor{cverbbg}{gray}{0.93}

\usepackage{pythontex}

\newcommand{\makesub}[1]{%
  \saveexpandmode\noexpandarg
  \StrSubstitute{#1}{\_}{_}[\temp]%
  \restoreexpandmode%
}

\newcommand{\target}[1]{%
  \makesub{#1}%
  \hypertarget{\temp}{}%
}

\newcommand{\attach}[1]{%
  \makesub{#1}%
  \hyperlink{\temp}{\emph{#1}}%
}
\newcommand{\codeattach}[1]{%
  \makesub{#1}%
  \hyperlink{\temp}{\code{#1}}%
}

\newcommand{\code}[1]{\colorbox{cverbbg}{\texttt{\StrSubstitute{#1}{_}{\_}}}}

\newcommand{\nota}[1]{\textbf{Nota}: #1}
\newcommand{\memo}[1]{\textbf{Memo}: #1}
\newcommand{\esempio}[1]{\textbf{Esempio}: #1}
\newcommand{\tbs}{\textbackslash}
\newcommand{\ghidra}{\emph{Ghidra}}
\newcommand{\ollydb}{\emph{OllyDB}}
\newcommand{\Null}{\code{NULL}}
\newcommand{\dll}{\emph{DLL}}
\newcommand{\api}{\emph{API}}
\newcommand{\key}[1]{\texttt{#1}}

\newcommand{\win}[0]{\textit{Windows}}
\newcommand{\linux}[0]{\textit{Linux}}

\newenvironment{myverb}
 {\SaveVerbatim{cverb}}
 {\endSaveVerbatim
  \flushleft\fboxrule=0pt\fboxsep=.5em
  \colorbox{cverbbg}{%
    \makebox[\dimexpr\linewidth-2\fboxsep][l]{\BUseVerbatim{cverb}}%
  }
  \endflushleft
}





\setcounter{tocdepth}{4}
\setcounter{secnumdepth}{4}
\setlength\parindent{0pt}
\begin{document}\sloppy
  
\title{
  \textbf{
    \emph{Esercizio preliminare}
  }
}  
\author{Luca Mastrobattista}
\date{}
\maketitle

\tableofcontents

\newpage
\section{Traccia}
Considerare il progetto \href{https://github.com/apache/bookkeeper}{Apache Bookkeeper}. In particolare la \href{http://bookkeeper.apache.org/docs/4.10.0/development/protocol/}{documentazione}, con riferimento alla classe \href{https://github.com/apache/bookkeeper/blob/e7430ce25e3609bf77b028e4f0475dad2147c22d/bookkeeper-server/src/main/java/org/apache/bookkeeper/client/BookKeeper.java}{\key{org.apache.bookkeeper.client.Bookkeeper}}, identificare partizioni di dominio per i parametri dei metodi: \key{createLedger}.\\

Suggerimento: iniziare a considerare \href{https://github.com/apache/bookkeeper/blob/e7430ce25e3609bf77b028e4f0475dad2147c22d/bookkeeper-server/src/main/java/org/apache/bookkeeper/client/BookKeeper.java#L887}{la sua implementazione}.\\
 
\noindent Scrivere un breve report descrivendo:
\begin{enumerate}
\item come sono state identificate le partizioni, e come è stata condotta la \textit{boundary analysis}
\item suggerimenti preliminari di test da includere, con risultato atteso
\end{enumerate}

\section{Svolgimento}
La segnatura del metodo è:
\begin{myverb}
public LedgerHandle createLedger(int ensSize, 
      int writeQuorumSize, 
      int ackQuorumSize, 
      DigestType digestType, 
      byte[] passwd)
\end{myverb}
Prende in inpunt 5 parametri:
\begin{enumerate}
\item \key{ensSize}: il numero di nodi in cui il \textit{ledger} è memorizzato. La segnatura prevede un tipo intero.

\item \key{writeQuorumSize}: il numero di nodi in cui ogni \textit{entry} è memorizzata e rappresenta di fatto il massimo grado di replicazione di ogni \textit{entry}. La segnatura prevede un tipo intero.

\item \key{ackQuorumSize}: il numero di nodi da cui ogni \textit{entry} deve essere conosciuta e rappresenta di fatto il minimo grado di replicazione di ogni \textit{entry}. Anche qui è previsto un intero.

\item \key{digestType}: dovrebbe rappresentare una \textit{signature} del \textit{ledger} calcolata sulle varie \textit{entry} per verificarle. È una \textit{enum} e sono possibili solo i seguenti valori:
	\begin{enumerate}
	\item \key{CRC32}
    \item \key{MAC}
 	\item \key{CRC32C}
    \item \key{DUMMY}
	\end{enumerate}

\item \key{passwd}: nella documentazione non è riportata l'utilità di questo campo, ma è possibile leggere che sono i bytes ottenuti da una password inserita. È quindi un array di bytes. 
\end{enumerate}

\subsection{Definizione delle classi di equivalenza}

\subsubsection{\key{ensSize}}
È un tipo di dato intero che rappresenta il numero di nodi su cui il \textit{ledger} è definito. Si definiscono 3 classi di equivalenza, una invalida, una che potrebbe essere invalida, e una valida:\\
\{< 0\}, \{= 0\}, \{> 0\}

\subsubsection{\key{writeQuorumSize}}
È un tipo di dato intero che rappresenta il numero di nodi su cui ogni \textit{entry} è memorizzata, e quindi il suo grado massimo di replicazione. Si definiscono 3 classi di equivalenza, una invalida, una che potrebbe essere invalida, e una valida:\\
\{< 0\}, \{= 0\}, \{> 0\}

\subsubsection{\key{ackQuorumSize}}
È un tipo di dato intero che rappresenta il numero di nodi su cui ogni \textit{entry} deve essere conosciuta, e quindi il suo grado minimo di replicazione. Si definiscono 3 classi di equivalenza, una invalida, una che potrebbe essere invalida, e una valida:\\
\{< \key{writeQuorumSize}\}, \{= \key{writeQuorumSize}\}, \{> \key{writeQuorumSize}\}

\subsubsection{\key{digestType}}
Si definisce una classe di equivalenza per ogni \textit{enum}:\\
\{\key{CRC32}\}, \{\key{MAC}\}, \{\key{CRC32C}\}, \{\key{DUMMY}\}

\subsubsection{\key{passwd}}
È un array di bytes. Si definiscono le seguenti classi:\\
\key{NULL}, \{\}, \key{"password".getBytes()}.


\subsection{Boundary analisys}
Le linee guida prevedono che si scelgano valori ai bordi delle classi di equivalenza dove, secondo uno studio empirico, si riscontrano più spesso problemi. 

\subsubsection{\key{ensSize}}
\{< 0\}, \{= 0\}, \{> 0\} $\rightarrow$ \{-1\}, \{0\}, \{1\}

\subsubsection{\key{writeQuorumSize}}
\{< 0\}, \{= 0\}, \{> 0\} $\rightarrow$ \{-1\}, \{0\}, \{1\}

\subsubsection{\key{ackQuorumSize}}
\{< \key{writeQuorumSize}\}, \{= \key{writeQuorumSize}\}, \{> \key{writeQuorumSize}\} $\rightarrow$ \{\key{writeQuorumSize} -1\}, \{\key{writeQuorumSize}\}, \{\key{writeQuorumSize} +1\}

\subsubsection{\key{digestType}}
\{\key{CRC32}\}, \{\key{MAC}\}, \{\key{CRC32C}\}, \{\key{DUMMY}\}

\subsubsection{\key{passwd}}
\key{NULL}, \{\}, \key{"password".getBytes()}


\subsection{Test da includere}
\newcommand{\f}[1]{\key{createLedger(#1)}}
\subsubsection{Analisi unidimensionale}
\f{-1, -1, -2, \key{CRC32}, \key{NULL}} $\rightarrow$ Ci si aspetta il sollevamento di una eccezione: si suppone che i gradi di replicazione di un nodo debbano essere maggiori o al massimo uguali a 0. \\

\f{0, 0, 0, \key{MAC}, \key{\{\}}} $\rightarrow$ Ci si aspetta il sollevamento di una eccezione: nella documentazione è specificato che è necessario almeno una password e un \key{digestType} per creare un ledger, ma qui la password è vuota. Inoltre non avrebbe molto senso creare un \textit{ledger} che non viene memorizzato su nessun nodo. \\

\f{1, 1, 2, \key{CRC32C}, \key{"password".getBytes()}} $\rightarrow$ Ci si aspetta il sollevamento di una eccezione: il grado di replicazione minimo è superiore al grado di replicazione massimo. \\

\f{1, 1, 2, \key{DUMMY}, \key{"password".getBytes()}} $\rightarrow$ Ci si aspetta che venga restituito un \key{LedgerHandle} correttamente creato. \\

L'ultimo test è risultato necessario per coprire interamente le classi di equivalenza del parametro \key{digestType}, ma è stato utile per testare che il metodo restituisca un oggetto correttamente inizializzato. 

\subsubsection{Analisi multidimensionale}
Le possibili combinazioni sono: 
\[
\prod_{C_i}{|C_i|} = 3 \cdot 3 \cdot 3 \cdot 4 \cdot 3 = 3^4 \cdot 4 = 324
\]
dove $C_i$ è la classe di equivalenza per il parametro $i$.

\subsubsection{Conclusioni}
Il secondo metodo, quello multidimensionale, è decisamente più costoso ma sicuramente più esaustivo: tutte le possibili combinazioni delle classi di equivalenza vengono infatti testate. Non è detto, però, che eventuali malfunzionamenti dipendano dalla combinazione di tutti e 5 i parametri: personalmente, osservando che i test definiti con un approccio unidimensionale sono un sottoinsieme dei casi definiti con un approccio multidimensione, proverei prima ad eseguire i test del sottoinsieme e poi, eventualmente, eseguirei i test mancanti per ottenere l'insieme dei test definiti nel caso multidimensionale. 

\end{document}